% Options for packages loaded elsewhere
\PassOptionsToPackage{unicode}{hyperref}
\PassOptionsToPackage{hyphens}{url}
\documentclass[
]{article}
\usepackage{xcolor}
\usepackage{amsmath,amssymb}
\setcounter{secnumdepth}{-\maxdimen} % remove section numbering
\usepackage{iftex}
\ifPDFTeX
  \usepackage[T1]{fontenc}
  \usepackage[utf8]{inputenc}
  \usepackage{textcomp} % provide euro and other symbols
\else % if luatex or xetex
  \usepackage{unicode-math} % this also loads fontspec
  \defaultfontfeatures{Scale=MatchLowercase}
  \defaultfontfeatures[\rmfamily]{Ligatures=TeX,Scale=1}
\fi
\usepackage{lmodern}
\ifPDFTeX\else
  % xetex/luatex font selection
\fi
% Use upquote if available, for straight quotes in verbatim environments
\IfFileExists{upquote.sty}{\usepackage{upquote}}{}
\IfFileExists{microtype.sty}{% use microtype if available
  \usepackage[]{microtype}
  \UseMicrotypeSet[protrusion]{basicmath} % disable protrusion for tt fonts
}{}
\makeatletter
\@ifundefined{KOMAClassName}{% if non-KOMA class
  \IfFileExists{parskip.sty}{%
    \usepackage{parskip}
  }{% else
    \setlength{\parindent}{0pt}
    \setlength{\parskip}{6pt plus 2pt minus 1pt}}
}{% if KOMA class
  \KOMAoptions{parskip=half}}
\makeatother
\setlength{\emergencystretch}{3em} % prevent overfull lines
\providecommand{\tightlist}{%
  \setlength{\itemsep}{0pt}\setlength{\parskip}{0pt}}
\usepackage{bookmark}
\IfFileExists{xurl.sty}{\usepackage{xurl}}{} % add URL line breaks if available
\urlstyle{same}
\hypersetup{
  hidelinks,
  pdfcreator={LaTeX via pandoc}}

\author{}
\date{}

\begin{document}

\textbf{Abstract (v2)}

This is not a neutral book. It is a systems science field manual forged in urgency. I wrote it not as an academic exercise, but as a call to arms against the structural entropy eroding our shared reality. Across fifteen chapters, I map the engineered collapse of democratic institutions, epistemic trust, and human dignity through the lens of complex adaptive systems. The framework is rigorous, but the threat is real.

We are not merely witnessing decay; we are complicit in a feedback loop of normalization. Disinformation isn't a glitch. It's a feature. The gamification of attention, the commodification of outrage, and the strategic use of noise to drown out signal- all maximize entropy in social systems. Using entropy models, Markov state transitions, and Nash equilibria, this book demonstrates how power operates not through transparency but through obfuscation and coercive incentives (Cover, 1999; Landauer, 1961; Shannon, 1948).

Erasure is energy-intensive. Drawing on Landauer's Principle, I show how the erasure of truth, memory, and civic history incurs a thermodynamic cost. As our systems overwrite inconvenient realities, they expend credibility and energy. Ultimately, this pattern is unsustainable. Entropy wins.

Each chapter builds a systemic scaffold for understanding collapse: from weaponized narrative (ch.~9), to coercive consent engineering (ch.~12), to the terminal feedback loops of crisis management (ch.~6). Rather than propose false hope, I offer systems awareness as resistance. To read this book is to become a vector for counter-coherence.

This work synthesizes systems theory, cognitive science, legal precedent, and open-source intelligence (OSINT) research. All models are grounded in empirical studies, historical analogs, and real-world events, including the judicial permissiveness that enables digital coercion and the algorithmic acceleration of social unraveling (Zuboff, 2019; Cohen, 2019; Page, 2010).

I offer this manuscript as a resistance node for those who refuse to sleep through systemic collapse. Every line was written with the conviction that understanding complexity is the first act of civic defiance. What comes next is up to you.

\textbf{Keywords:} Systems Collapse, Entropy, Disinformation, Consent Engineering, Thermodynamic Cost, Feedback Loops, Complex Systems, Landauer Limit, Nash Equilibrium, Civic Resistance

\textbf{References:}
Cohen, J. E. (2019). \emph{Between Truth and Power: The Legal Constructions of Informational Capitalism}. Oxford University Press.\\
Cover, T. M., \& Thomas, J. A. (1999). \emph{Elements of Information Theory}. Wiley.\\
Landauer, R. (1961). Irreversibility and heat generation in the computing process. \emph{IBM Journal of Research and Development}, 5(3), 183--191.\\
Page, S. E. (2010). \emph{Diversity and Complexity}. Princeton University Press.\\
Shannon, C. E. (1948). A Mathematical Theory of Communication. \emph{Bell System Technical Journal}, 27, 379--423.\\
Zuboff, S. (2019). \emph{The Age of Surveillance Capitalism}. PublicAffairs.

\end{document}
