\section{Chapter 6: Crisis Loops and the Logic of
Self-Destruction}\label{chapter-6-crisis-loops-and-the-logic-of-self-destruction}

\subsection{The System Eats Itself}\label{the-system-eats-itself}

No complex system collapses all at once---it feeds on its own feedback
until it implodes. Crisis becomes the new equilibrium. In America, we
are witnessing crisis loops---self-perpetuating feedback cycles that no
longer aim to solve problems but to exploit them for profit and power.
If democracy were a living organism, it would now be devouring its own
tissue in search of calories.

Consider the political economy of disaster: when calamity strikes,
whether it's a pandemic, mass shooting, climate event, or insurrection,
certain actors profit. Corporations sell ``solutions,'' politicians
fundraise off fear, and the media monetizes outrage (Zuboff, 2019). The
system has no incentive to break the loop because the loop is lucrative.
In systems terms, this is positive feedback run amok---growth not toward
equilibrium, but toward instability.

The more we spin inside these crisis loops, the harder it becomes to
intervene. Each new shock is absorbed by a system trained to weaponize
dysfunction. Institutional legitimacy erodes, while a kind of disaster
fatigue sets in among the public. As a result, accountability dies.

\subsection{The Political Utility of
Collapse}\label{the-political-utility-of-collapse}

Collapse, far from being avoided, becomes politically useful. In 2025,
political operatives openly discussed the benefits of destabilization:
reduced regulatory barriers, easier judicial manipulation, and a
population too distracted or frightened to resist (Chotiner, 2025). When
the system rewards sabotage, sabotage becomes the system.

A powerful example lies in the judicial system. The Supreme Court's
alignment with MAGA priorities is no longer hidden. As Ms.~Maine (2025)
observes, it increasingly serves not as a neutral arbiter, but as a
mechanism of exclusion and cruelty. Its rulings now reflect a deliberate
entrenchment of minority rule.

The structural fragility that results is systemic, not accidental. And
it radiates outward: through economic inequality, information warfare,
and climate breakdown. Each crisis begets another, and each is
metabolized into political opportunity by those who engineer the system.

\subsection{Thermodynamic Collapse: The Cost of Erasing
Truth}\label{thermodynamic-collapse-the-cost-of-erasing-truth}

Landauer's Principle tells us that erasing one bit of information costs
energy (Bérut et al., 2012). That may seem trivial---until you apply it
at scale. What happens when entire segments of reality are erased? When
governments wipe data, when platforms censor history, when regimes
rewrite memory? The energy cost isn't just computational---it's social
and cognitive. Whole institutions are destabilized.

In the context of American disinformation, every deletion of
truth---about history, science, or democracy---creates an energetic
burden. The public must work harder to recover what was lost, if they
ever can. The result: exhaustion, confusion, apathy. Collapse.

The concept extends beyond physics: it enters the realm of epistemology
and civic integrity. Systems that erase history must expend massive
effort to maintain that erasure. Eventually, they fail. The more they
erase, the more energy they need, until they collapse under their own
informational contradictions.

\subsection{A Layperson's Framework}\label{a-laypersons-framework}

Imagine a house where every memory, every truth, every lesson must be
consciously preserved or it vanishes. Now imagine that house has no
books, no records, no pictures. The residents must keep everything in
their heads. As time passes, they forget---then repeat their mistakes.
This is America under engineered crisis loops: memory-holed, exhausted,
and reactive.

Landauer's Principle reminds us that memory has a cost. And losing
memory has a greater one. Every time a system erases truth, history, or
memory, it pays for it in energy and complexity---and ultimately
collapses.

\subsection{References}\label{references}

Bérut, A., Arakelyan, A., Petrosyan, A., Ciliberto, S., Dillenschneider,
R., \& Lutz, E. (2012). \emph{Experimental verification of Landauer's
principle linking information and thermodynamics}. Nature, 483(7388),
187--189.

Hartzog, W., \& Selinger, E. (2020). \emph{Privacy's Blueprint: The
Battle to Control the Design of New Technologies}. Harvard University
Press.

Penney, J. W. (2017). \emph{Chilling Effects: Online Surveillance and
Wikipedia Use}. Berkeley Technology Law Journal, 31(1), 117--182.

Zuboff, S. (2019). \emph{The Age of Surveillance Capitalism: The Fight
for a Human Future at the New Frontier of Power}. PublicAffairs.

Chotiner, I. (2025, May 16). \emph{Donald Trump's Culture of
Corruption}. The New Yorker.

Maine, M. (2025, June 23). \emph{When the Supreme Court Joins the MAGA
Regime}. Substack.
