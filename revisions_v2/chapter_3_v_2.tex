% Options for packages loaded elsewhere
\PassOptionsToPackage{unicode}{hyperref}
\PassOptionsToPackage{hyphens}{url}
\documentclass[
]{article}
\usepackage{xcolor}
\usepackage{amsmath,amssymb}
\setcounter{secnumdepth}{-\maxdimen} % remove section numbering
\usepackage{iftex}
\ifPDFTeX
  \usepackage[T1]{fontenc}
  \usepackage[utf8]{inputenc}
  \usepackage{textcomp} % provide euro and other symbols
\else % if luatex or xetex
  \usepackage{unicode-math} % this also loads fontspec
  \defaultfontfeatures{Scale=MatchLowercase}
  \defaultfontfeatures[\rmfamily]{Ligatures=TeX,Scale=1}
\fi
\usepackage{lmodern}
\ifPDFTeX\else
  % xetex/luatex font selection
\fi
% Use upquote if available, for straight quotes in verbatim environments
\IfFileExists{upquote.sty}{\usepackage{upquote}}{}
\IfFileExists{microtype.sty}{% use microtype if available
  \usepackage[]{microtype}
  \UseMicrotypeSet[protrusion]{basicmath} % disable protrusion for tt fonts
}{}
\makeatletter
\@ifundefined{KOMAClassName}{% if non-KOMA class
  \IfFileExists{parskip.sty}{%
    \usepackage{parskip}
  }{% else
    \setlength{\parindent}{0pt}
    \setlength{\parskip}{6pt plus 2pt minus 1pt}}
}{% if KOMA class
  \KOMAoptions{parskip=half}}
\makeatother
\setlength{\emergencystretch}{3em} % prevent overfull lines
\providecommand{\tightlist}{%
  \setlength{\itemsep}{0pt}\setlength{\parskip}{0pt}}
\usepackage{bookmark}
\IfFileExists{xurl.sty}{\usepackage{xurl}}{} % add URL line breaks if available
\urlstyle{same}
\hypersetup{
  hidelinks,
  pdfcreator={LaTeX via pandoc}}

\author{}
\date{}

\begin{document}

\textbf{Chapter 3: Systems in Decay (v2)}

We are living through a slow-motion collapse. This isn't hyperbole. It's the inevitable result of a system that has been disassembling itself from the inside. Institutions that once ensured stability now serve as amplifiers of division. Feedback loops, once designed to maintain equilibrium, have been hijacked by actors who profit from disorder. In this chapter, I will show how the structures of governance, civic trust, and public discourse have degraded into vectors of instability---and why this isn't a bug in the system, but its new operating logic.

\subsubsection{The Fracturing of Institutional Integrity}\label{the-fracturing-of-institutional-integrity}

The United States once relied on institutions to create a stabilizing center. Congress, the judiciary, and federal agencies were imperfect, but they provided checks against authoritarian impulse. Today, these same institutions are hollowed out---vulnerable to bad faith actors who exploit their procedures for personal gain (Levitsky \& Ziblatt, 2018). Congressional hearings have become partisan theater. The Supreme Court, now firmly aligned with a political agenda, upholds decisions that contradict public opinion and legal precedent (Chotiner, 2025).

Whistleblowers are criminalized. Inspectors general are fired or ignored. Agency leadership is packed with loyalists rather than experts. What remains is institutional theater: the appearance of process masking the collapse of substance.

\subsubsection{Cognitive Capture and Epistemic Decay}\label{cognitive-capture-and-epistemic-decay}

One of the most insidious forms of system decay is cognitive capture: the process by which a population's shared understanding of truth is eroded. We now live in fractured information ecosystems. The media landscape has splintered into polarized spheres, where algorithmic feeds serve confirmation bias rather than challenge it (Haidt \& Rose-Stockwell, 2021).

Disinformation thrives in such an environment. The same system that once produced informed citizens now rewards outrage, virality, and simplistic narratives. As a result, we've replaced nuanced debate with moral panic, evidence with vibes. Political actors like Donald Trump have capitalized on this collapse, feeding conspiracies and lies into the public sphere with impunity (Snyder, 2017).

This isn't a passive failure. It's an engineered condition. Cognitive capture is a strategy: disorient the public, erode consensus, and make truth a partisan commodity.

\subsubsection{Surveillance Capitalism and Democratic Degradation}\label{surveillance-capitalism-and-democratic-degradation}

The commodification of behavior and belief has enabled a new form of extraction. Surveillance capitalism, as Zuboff (2019) describes it, is not just a business model. It's a form of control. When every click, share, or scroll becomes a data point, platforms can nudge entire populations toward apathy or extremism.

This level of manipulation erodes civic agency. It disables the capacity for critical reflection. And it collapses the boundary between public interest and private profit. In such a system, voters become data sets. Citizens become users. Democracy becomes an illusion.

\subsubsection{Feedback Loops in Crisis}\label{feedback-loops-in-crisis}

System science tells us that feedback loops can be stabilizing or destabilizing. Positive feedback loops amplify change; negative loops resist it. What we see today is the predominance of runaway positive loops: outrage amplifies more outrage, polarization intensifies, and misinformation spreads faster than correction (Page, 2011).

The system is no longer self-correcting. In fact, its internal signals have been distorted beyond recognition. The feedback loops we now inhabit are tuned for acceleration---toward extremism, inequality, and fragmentation.

\subsubsection{Judicial Capture and Legal Incoherence}\label{judicial-capture-and-legal-incoherence}

The judiciary---once the last firewall against authoritarian overreach---has become an accelerant. Recent rulings not only contradict democratic norms but also dismantle precedent with strategic intent. Cases like \emph{Dobbs v. Jackson} and the 2025 decision on third-country deportations illustrate a Court that no longer interprets law, but engineers ideology (Maine, 2025).

We are watching the institutionalization of cruelty. Legal mechanisms are being wielded not to protect rights, but to redraw the boundaries of who qualifies as a rights-bearing subject.

This isn't law. It's legal theater in service of regime goals.

\subsubsection{Disintegration of Civic Trust}\label{disintegration-of-civic-trust}

When systems decay, people disengage. Civic participation plummets. Belief in democratic institutions collapses. But what emerges is not apathy---it's cynicism weaponized by authoritarians.

Populists like Trump understand this dynamic well. They don't need to win trust. They only need to convince you that \emph{no one} is trustworthy. When the system appears broken beyond repair, authoritarianism begins to look like order. And that illusion is enough to tip the scales.

\subsubsection{The System is Working as Designed---But for Whom?}\label{the-system-is-working-as-designedbut-for-whom}

Many ask: why aren't the safeguards working? The answer is uncomfortable: they \emph{are} working---just not for democracy. They have been retooled to protect power, not distribute it. What we are seeing isn't a failure of the system, but its success in serving new masters.

Systems don't collapse evenly. They decay at the margins first: for the poor, the racialized, the undocumented. But eventually, the rot reaches the core. What was once invisible becomes unavoidable.

The task now is twofold: to understand this decay not as episodic, but as systemic; and to begin building feedback systems that reward truth, distribute power, and resist authoritarian logic.

This is no longer reform. It is reconstruction.

\begin{center}\rule{0.5\linewidth}{0.5pt}\end{center}

\subsubsection{References (APA 7th Edition)}\label{references-apa-7th-edition}

Chotiner, I. (2025, May 16). \emph{Donald Trump's Culture of Corruption}. The New Yorker. https://www.newyorker.com/news/q-and-a/donald-trumps-culture-of-corruption

Haidt, J., \& Rose-Stockwell, T. (2021). \emph{Why the Past 10 Years of American Life Have Been Uniquely Stupid}. The Atlantic.

Levitsky, S., \& Ziblatt, D. (2018). \emph{How Democracies Die}. Crown.

Maine, M. (2025, June 23). \emph{When the Supreme Court Joins the MAGA Regime: It's Time to Dismantle the System}. Substack.

Page, S. E. (2011). \emph{Diversity and Complexity}. Princeton University Press.

Snyder, T. (2017). \emph{On Tyranny: Twenty Lessons from the Twentieth Century}. Tim Duggan Books.

Zuboff, S. (2019). \emph{The Age of Surveillance Capitalism}. PublicAffairs.

\end{document}
