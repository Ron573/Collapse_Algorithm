\begin{flushleft}
\vspace*{1cm}
\textbf{\large Epigraph}

\vspace{1.5em}
\begin{quote}\raggedright\itshape
In 1963, Dr. Martin Luther King Jr. wrote from a Birmingham jail that “human progress never rolls in on wheels of inevitability.”

In 2025, I believe the same urgency applies. Human progress and American democracy will not continue by default. They must be defended.

Now is our time to stand guard at home — with grim and bold determination, yet with a clarity of purpose that rises above vengeance.
\end{quote}
\hfill --- Botelho, R. (2025), \textit{Guarding the Fragile Architecture of Democracy}

\vspace{2em}
\hrule
\vspace{1em}

\begin{quote}\raggedright\itshape
To erase a single bit of information, a physical system must dissipate energy — this is not a metaphor, but a law of thermodynamics.

Landauer’s Principle sets the fundamental energy cost of information erasure. The closer one approaches zero temperature, the more severe the cost. There is no clean slate — only entropy, displaced.

This book begins with that act of erasure. Because before systems collapse outwardly, they unravel from within.
\end{quote}
\hfill --- Adapted from Rolf Landauer (1961) and extended by Botelho, R. (2025)
\end{flushleft}

