\chapter*{Preface}
\addcontentsline{toc}{chapter}{Preface}

This book was born of unease — not in theory, but in practice. Something fundamental had shifted in the structure of our systems. What once felt like gradual decay became something else: collapse in motion, masked by noise, and made palatable by euphemism.

I did not write this because I wanted to.

I wrote it because I had to.

For years, I watched as entropy crept into the seams of our institutions. Language was weaponized. Feedback loops failed. The truth became a matter of alignment, not accuracy. Patterns once relegated to historical outliers were returning with systemic force — and this time, we were the ones inside the spiral.

Each chapter here is a reckoning.

I have drawn from systems science, thermodynamics, information theory, and personal experience — not for abstraction, but to anchor what we all feel yet cannot always name. Collapse is rarely declared. It is felt in a thousand small frictions: in eroded trust, in deferred accountability, in the slow dismemberment of reality.

This is not a book of despair.

It is a map of where we are, how we got here, and how — just maybe — we can resist the slide toward irreversible failure. I offer no illusions. But I do offer a frame: if collapse is a process, then so is repair.

Let’s begin with that hope.

\begin{flushright}
Ronald J. Botelho, MS\\
Binghamton University\\
2025
\end{flushright}

